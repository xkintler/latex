	\begin{table}[h]
	\centering
	\caption{Použitie Eulerovho vzorca pre všetky pravidelné mnohosteny.}
	\label{tab 1}
	\begin{tabular}{|c|c|c|c|}
		\hline
		\multirow{2}{2.5cm}{\centering \textbf{Pravidelny mnohosten}}&\multirow{2}{3cm}{\centering \textbf{V - E + F = 2}}&\multirow{2}{2cm}{\centering \textbf{Vysledok}}&\multirow{2}{2cm}{\centering \textbf{Platnost vzorca}}\\ 
		&&& \\ \hline
		Strvorsten&$4 - 6 + 4 = 2$&$2 = 2$&{\color{green}plati} \\ \hline
		Seststen&$8 - 12 + 6 = 2$&$2 = 2$&{\color{green}plati} \\ \hline
		Osemsten&$6 - 12 + 8 = 2$&$2 = 2$&{\color{green}plati} \\ \hline
		Dvanaststen&$20 - 30 + 12 = 2$&$2 = 2$&{\color{green}plati} \\ \hline
		Dvanaststen&$12 - 30 + 20 = 2$&$2 = 2$&{\color{green}plati} \\ \hline
	\end{tabular}
\end{table}
\noindent
Z uvedených výsledkov v tabuľke 1 vyplýva, že Eulerov vzorec platí pre všetky uvedené pravidelné mnohosteny. V nasledujúcich analýzach pravidelných mnohostenov môžme teda vychádzať z toho, že jedna z podmienok, ktoré musí pravidelný mnohosten spĺňať je práve platnosť Eulerovho vzorca.